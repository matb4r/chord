\documentclass[12pt, twoside, openany]{report}
\usepackage[dvips]{graphicx,color,rotating}
\usepackage[utf8]{inputenc}
\usepackage{indentfirst}
\usepackage{t1enc}
\usepackage{a4wide}
\usepackage{amsfonts}
\usepackage{amsmath}
\usepackage{caption}
\usepackage{enumerate}
\usepackage{verbatim}
\usepackage[MeX]{polski}
\usepackage[T1]{fontenc}
\usepackage{geometry}
\geometry{left=25mm,right=25mm,bindingoffset=10mm, top=25mm, bottom=25mm}
\usepackage{amssymb, latexsym}
\usepackage{amsthm}
\usepackage{palatino}
\usepackage{float}
\usepackage{tabulary}
\usepackage{array}
\usepackage{pstricks}
\usepackage{textcomp}
\usepackage{url}
\usepackage{listings}
\usepackage{xcolor}
\usepackage{courier}
\usepackage{pgfplots}
\usepackage{etoolbox}
\patchcmd{\abstract}{\null\vfil}{}{}{} % abstract top align.

\newtheorem{twierdzenie}{Twierdzenie}[section]

%\linespread{1.5}

% styl listingów
\lstset{
    xleftmargin=17pt,
    numbers = left,
    %framexleftmargin=10mm,
    frame=lrbt,
    %backgroundcolor=\color[RGB]{250,250,250},
    %keywordstyle=\bfseries\color{blue},
    %identifierstyle=\bfseries,
    %numberstyle=\color[RGB]{0,192,192},
    %commentstyle=\it\color[RGB]{96,96,96},
    %stringstyle=\rmfamily\slshape\color[RGB]{128,0,0},
    %showstringspaces=true,
    breaklines=true,
    breakatwhitespace=try,
    %language=C++,
    showstringspaces=false,
    tabsize=4,
    basicstyle=\footnotesize\ttfamily,
    aboveskip=1em
    %postbreak=\raisebox{0ex}[0ex][0ex]{\ensuremath{\color{red}\hookrightarrow\space}}
}


\author{Mateusz Bartkowiak}
\title{Współbieżne modelowanie i~prezentacja dynamiki obiektów fizycznych}


\begin{document}

\begin{titlepage}

\noindent


\centering
\Large POLITECHNIKA POZNAŃSKA\\
\Large Wydział Informatyki\\
\large Systemy Rozproszone

\vfill
\includegraphics[width=120pt,height=120pt]{PP}

\vfill
\center
\LARGE
Mateusz Bartkowiak

\center
\Large
Praca magisterska

\vfill
\center
\Huge
\textbf{Efektywne wyszukiwanie danych w systemach P2P w oparciu o rozszerzenia protokołu Chord}


\vfill
\center
\Large
Promotor: dr inż. Anna Kobusińska

\vfill
\center
\large
Poznań, Październik 2018

\end{titlepage}

% wlasny abstract
\renewcommand{\abstractname}{}
\begin{abstract}
\thispagestyle{plain} % numer strony
\begin{center}
\textbf{Streszczenie}
\end{center}

\indent
Polskie streszczenie pracy.

\vspace{50px}
\begin{center}
\textbf{Abstract}
\end{center}

\indent
English abstract

\end{abstract}


\tableofcontents


%-----------Poczatek czesci zasadniczej-----------

\chapter{Wstęp}

\begin{itemize}
\item cel pracy i motywacja
\item zakres i omówienie rozdziałów
\end{itemize}


\chapter{Systemy Peer-to-Peer}

System P2P (ang. \textit{Peer-to-Peer}) jest z definicji systemem w którym wszystkie węzły tworzące sieć są równoważne w sensie funkcjonalnym i pełnią rolę zarówno klienta jak i serwera. Każdy węzeł może bezpośrednio nawiązywać połączenie z innym. Brak jest centralnego serwera który np. byłby głównym magazynem danych, bądź pośredniczyłby w komunikacji. Założenia te rodzą wiele problemów i ograniczeń, dlatego też budując komercyjne systemy P2P często odchodzi się od tej definicji wyróżniając specjalne węzły, które pełnią dodatkowe funkcje. Rozdział ten opisuje różne budowy systemów P2P, pojawiające się problemy i wyzwania które należy podjąć, oraz przedstawia już istniejące i wykorzystywane rozwiązania.

\section{Sieć nakładkowa}
Systemy P2P wykorzystują do działania sieć nakładkową (ang. \textit{overlay network}), która jest dodatkową warstwą abstrakcji nadbudowaną nad istniejącą siecią (np. Internet). Parametry niefunkcjonalne systemu P2P takie jak odporność na awarie, autonomia węzłów, wydajność, skalowalność czy bezpieczeństwo zależą w dużej mierze od budowy sieci nakładkowej, którą można podzielić na trzy główne typy: nieustrukturyzowane, ustrukturyzowane oraz sieci z tzw. super-węzłami (ang. \textit{super-peer network}).

W sieciach nieustrukturyzowanych, sieć nakładkowa jest budowana ad hoc w sposób niedeterministczny. Rozmieszczenie danych jest zupełnie niezależne od topologii sieci nakładkowej, a każdy węzeł wie jedynie o swoich sąsiadach, aczkolwiek nie jest poinformowany o zasobach jakie posiadają. Mechanizmy wyszukiwania są z reguły proste, acz kosztowne. Przykładem może być mechanizm zalewania sieci zapytaniami o dany zasób, które krążą po sieci dopóki żądany zasób nie zostanie odnaleziony. Innym, nieco bardziej wyrafinowanym i wydajnym sposobem jest przesyłanie kilku równoległych zapytań, które każde jest przesyłane między węzłami w taki sposób, że jeden węzeł przesyła dalej zapytanie tylko do swojego jednego sąsiada. Z tego typu sieci nakładkowej korzystają m.in. takie systemy P2P jak Gnutella, Kazaa, czy FreeHaven. Jednym z największch wad sieci nieustrukturalizowanych jest niska skalowalności, gdyż przy coraz to większej ilości węzłów w sieci, znalezienie danego zasobu jest coraz bardziej kosztowne. Odpowiedzią na ten problem jest sieć ustrukturalizowana.
% mozna dopisac wiecej przykladow mechanizmow z artykułu, mozna pokazac ze dziela sie na homogeniczne, heterogeniczne i zcentralizowane

Sieci ustrukturyzowane charakteryzują się uzależnieniem lokalizacji danych od topologii sieci. Każdy zasób posiada swój identyfikator, który jednocześnie wskazuje miejsce w sieci, w którym się znajduje. Głównym reprezentantem tej klasy sieci są systemy DHT (ang. \textit{distributed hash table}). Systemy te udostępniają interfejs tablic mieszających, gdzie klucze są identyfikatorami obiektów. Każdy węzeł odpowiedzialny jest za przechowywanie tych danych, których wartość kluczy mieści się w odpowiednim dla niego, ściśle określonym zakresie. Ponadto każdy węzeł ma informacje o pewnej liczbie innych węzłów w sieci (sąsiadów): przechowuje tablice rutingu w której przypisane są identyfikatory sąsiadów do odpowiednich adresów. Większość operacji dostępu do danych to operacje \textit{lookup}, czyli próby znalezienia lokalizacji jakiegoś obiektu. Operacja lookup najczęściej wykorzystywana jest do znalezienia adresu węzła odpowiedzialnego za dany zasób, dzięki czemu węzeł pytający może bezpośrednio nawiązać z nim komunikację. Aby skutecznie odnaleźć węzeł odpowiedzialny, pojedyńcze wywołanie tej operacji może skutkować kilkoma przeskokami żądania pomiędzy sąsiadami. Ponieważ każdy węzeł jest odpowiedzialny za pewien zakres kluczy oraz współtworzy system rutowania, autonomia pojedyńczego węzła jest mocno ograniczona. Jest to największą wadą sieci ustrukturalizowanych, gdyż każde dołączenie, bądź rozłączenie węzła zaburza strukturę sieci, co skutkuje potrzebą przeprowadzenia jej rekonstrukcji. Przykładami tego typu sieci są takie systemy jak CAN, Tapestry, Pastry, Pier, OceanStore, Past, czy protokół Chord, na rozszerzeniu którego opiera się niniejsza praca (więcej o protokole Chord w rozdziale \ref{rozdzial_chord}).

W sieciach ustrukturalizowanych oraz nieustrukturalizowanych wszystkie węzły są takie same pod względem funkcjonalności. Inaczej się ma sprawa z sieciami super-peer (sieci z super-węzłami), które stanowią klasę pośrednią, między modelem P2P a modelem klient-serwer. W sieciach tej klasy niektóre węzły pełnią specjalne role w sieci, spełniając takie dodatkowe funkcje jak indeksowanie, przetwarzanie zapytań, kontrola dostępu, czy zarządzanie metadanymi. W przypadku awarii, rolę super-węzłów mogą dynamicznie przejmować inne węzły. Działanie tych sieci przeważnie polega na wysyłaniu zapytań do super-węzła, który posiada informacje o położeniu żądanych obiektów. Dlatego też zaletami sieci super-peer są wydajność oraz jakość usługi (ang. \textit{quality of service}). Czas potrzebny na znalezienie żądanego obiektu jest zazwyczaj krótszy, niż w przypadku zalewania sieci. Ponadto heterogeniczność węzłów w sensie ich zasobów i wydajności można wykorzystać do wyróżnionych zadań, obierając je za super-węzły. Jednakże autonomia węzłów w owych sieciach jest ograniczona, a odporność na awarie jest niska, gdyż super-węzły stają się pojedyńczymi punktami awarii (skutki tego problemu można załagodzić dynamicznie wybierając nowe super-węzły). Z tego typu sieci korzystają m.in. takie systemy jak Napster, Publius, Edutella, czy JXTA.

\section{Problemy i wyzwania}

Każdy system dużej skali stoi przed problemem awarii różnych części systemu. W odróżnieniu od wysoce niezawodnych centrów danych, systemy P2P często składają się z niepewnych wezłów, które nieraz pracują w niestabilnych warunkach (np. fizycznie poruszający się węzeł będący podłączony do sieci przez interfejs bezprzewodowy), czy dołączają do sieci tylko na pewien okres czasu. Zjawisko częstego rozłączania wezłów z sieci nazwijmy \textit{odpływem}.

Systemy Peer-to-Peer są formą systemów rozproszonych z tą różnicą, że efekt odpływu jest w nich częstym i naturalnym zjawiskiem, co rodzi szereg dodatkowych problemów. Do tego dochodzą trudności typowe dla systemów rozproszonych, co razem sprawia, że skonstruowanie dobrego, w sensie poprawności i efektywności, systemu P2P jest zadaniem trudnym.

Dobry system P2P powinien spełniać szereg własności. Wybrane własności ważne z perspektywy zarządzania danymi są następujące:
\begin{itemize}
\item Dostepność: węzły powinny w każdym momencie mieć dostęp do potrzebnych danych.
\item Autonomia: węzły powinny móc dołączać i rozłączać się z sieci w każdej chwili, nie tworząc przy tym dodatkowych problemów, takich jak ograniczanie dostępności do danych.
\item Wydajność: system powinien efektywnie wykorzystywać dostępne zasoby sieci (przepustowość, moc obliczeniowa czy przechowywanie danych).
\item Jakość usług: z perspektywy użytkownika system powinien być wysokiej jakości, tzn. wyniki winne być kompletne, dane spójne i zawsze dostępne, a czas odpowiedzi powinien być krótki.
\item Odporność na awarie: ewentualne awarie, które się zdarzają, nie powinny mieć wpływu na wydajność, czy jakość usługi.
\item Bezpieczeństwo: naturalna otwartość systemów P2P sprawia, że bezpieczeństwo, głównie w kontekście nieuprawnionego dostępu do danych, jest dużym wyzwaniem.
\end{itemize}


\section{Replikacja}




\section{Istniejące rozwiązania}


\chapter{Protokół Chord}
\label{rozdzial_chord}

\begin{itemize}
\item idea
\item jak dziala
\item pseudokod
\item istniejące rozszerzenia
\end{itemize}


\chapter{Grupowanie węzłów}

\begin{itemize}
\item idea
\item istniejące rozwiązania
\end{itemize}


\chapter{Zaproponowane rozwiązanie}

\begin{itemize}
\item ogolny opis
\item do czego dążę
\item od czego abstrachuję
\item szczegolowy opis, pseudokod
\end{itemize}


\chapter{Testy symulacyjne}

\begin{itemize}
\item opisac PeerSim
\item scenariusze: ile wezlow, jak czesto dolaczaja, jak duze sa grupy, wspolczyniki stabilizacji
\item wyniki
\item analiza
\end{itemize}


\chapter{Podsumowanie}

\begin{itemize}
\item o czym byla praca, czy cel osiagniety, czy spoko wyniki, dalsze rozwiazania
\end{itemize}


%-----------Koniec czesci zasadniczej-----------

\begin{thebibliography}{11}
\addcontentsline{toc}{chapter}{Bibliografia} % dodane do spisu treści

\bibitem{bib:martins} Vidal Martins, \emph{Data Replication in P2P Systems}, Réseaux et télécommunications [cs.NI]. Université de Nantes, 2007. Français. <tel-00481828>

\bibitem{bib:kobusinska} Anna Kobusińska, \emph{Systemy Rozprosozne Dużej Skali}, Wykłady, [online] \url{http://www.cs.put.poznan.pl/akobusinska/lsds.html} [dostęp: 23.08.2018 r.]

\bibitem{bib:gpu_gems} Lars Nyland, Mark Harris, Jan Prins, \emph{GPU Gems 3}, Chapter 31. Fast N-Body Simulation with CUDA, [online] \url{http://http.developer.nvidia.com/GPUGems3/gpugems3_ch31.html} [dostęp: 01.01.2017]

\end{thebibliography}

\end{document}
