\documentclass[12pt, twoside, openany]{report}
\usepackage[dvips]{graphicx,color,rotating}
\usepackage[utf8]{inputenc}
\usepackage{indentfirst}
\usepackage{t1enc}
\usepackage{a4wide}
\usepackage{amsfonts}
\usepackage{amsmath}
\usepackage{caption}
\usepackage{enumerate}
\usepackage{verbatim}
\usepackage[MeX]{polski}
\usepackage[T1]{fontenc}
\usepackage{geometry}
\geometry{left=25mm,right=25mm,bindingoffset=10mm, top=25mm, bottom=25mm}
\usepackage{amssymb, latexsym}
\usepackage{amsthm}
\usepackage{palatino}
\usepackage{float}
\usepackage{tabulary}
\usepackage{array}
\usepackage{pstricks}
\usepackage{textcomp}
\usepackage{url}
\usepackage{listings}
\usepackage{xcolor}
\usepackage{courier}
\usepackage{pgfplots}
\usepackage{etoolbox}
\patchcmd{\abstract}{\null\vfil}{}{}{} % abstract top align.

\newtheorem{twierdzenie}{Twierdzenie}[section]

%\linespread{1.5}

% styl listingów
\lstset{
    xleftmargin=17pt,
    numbers = left,
    %framexleftmargin=10mm,
    frame=lrbt,
    %backgroundcolor=\color[RGB]{250,250,250},
    %keywordstyle=\bfseries\color{blue},
    %identifierstyle=\bfseries,
    %numberstyle=\color[RGB]{0,192,192},
    %commentstyle=\it\color[RGB]{96,96,96},
    %stringstyle=\rmfamily\slshape\color[RGB]{128,0,0},
    %showstringspaces=true,
    breaklines=true,
    breakatwhitespace=try,
    %language=C++,
    showstringspaces=false,
    tabsize=4,
    basicstyle=\footnotesize\ttfamily,
    aboveskip=1em
    %postbreak=\raisebox{0ex}[0ex][0ex]{\ensuremath{\color{red}\hookrightarrow\space}}
}


\author{Mateusz Bartkowiak}
\title{Współbieżne modelowanie i~prezentacja dynamiki obiektów fizycznych}


\begin{document}

\begin{titlepage}

\noindent


\centering
\Large POLITECHNIKA POZNAŃSKA\\
\Large Wydział Informatyki\\
\large Systemy Rozproszone

\vfill
\includegraphics[width=120pt,height=120pt]{PP}

\vfill
\center
\LARGE
Mateusz Bartkowiak

\center
\Large
Praca magisterska

\vfill
\center
\Huge
\textbf{Efektywne wyszukiwanie danych w systemach P2P w oparciu o rozszerzenia protokołu Chord}


\vfill
\center
\Large
Promotor: dr inż. Anna Kobusińska

\vfill
\center
\large
Poznań, Październik 2018

\end{titlepage}

% wlasny abstract
\renewcommand{\abstractname}{}
\begin{abstract}
\thispagestyle{plain} % numer strony
\begin{center}
\textbf{Streszczenie}
\end{center}

\indent
Polskie streszczenie pracy.

\vspace{50px}
\begin{center}
\textbf{Abstract}
\end{center}

\indent
English abstract

\end{abstract}


\tableofcontents


%-----------Poczatek czesci zasadniczej-----------

\chapter{Wstęp}

\begin{itemize}
\item uzasadniać temat pracy
\item ogólny opis obszaru - p2p
\item motywacja - dlaczego chce robic grupy?
\item cel i zakres
\item omówienie rozdziałów
\end{itemize}



\chapter{Przegląd istniejących rozwiązań}

\begin{itemize}
\item wspomniec o chordzie
\item opisac rozszerzenia chorda, raczej w kontekscie mojej pracy
\item moze zawrzec stabilizacje?
\end{itemize}


\chapter{Model systemu}

\begin{itemize}
\item opisac konkretnie, ustruktyryzowany/nieustr, jak sa wezly polaczone
\item wezly polaczone sa w logiczny pierscien…
\item mozna cos o grupach, ale bardzo oglnie, nie wchodzic w moje rozwiazanie
\end{itemize}


\chapter{Protokół Chord}

\begin{itemize}
\item jak dziala
\item pseudokod
\end{itemize}


\chapter{Zaproponowane rozwiązanie}

\begin{itemize}
\item ogolny opis, opisac od czego abstrachuje, moze stabilizacje zawrzec?
\item szczegolowy opis, pseudokod
\end{itemize}


\chapter{Testy symulacyjne}

\begin{itemize}
\item opisac PeerSim
\item scenariusze: ile wezlow, jak czesto dolaczaja, jak duze sa grupy, wspolczyniki stabilizacji
\item wyniki
\item analiza
\end{itemize}


\chapter{Podsumowanie}

\begin{itemize}
\item o czym byla praca, czy cel osiagniety, czy spoko wyniki, dalsze rozwiazania
\end{itemize}


%-----------Koniec czesci zasadniczej-----------

\begin{thebibliography}{11}
\addcontentsline{toc}{chapter}{Bibliografia} % dodane do spisu treści

\bibitem{bib:dziubak} Jacek Matulewski, Tomasz Dziubak, Marcin Sylwestrzak, Radosław Płoszajczak, \emph{Grafika, Fizyka, Metody numeryczne. Symulacje fizyczne z wizualizacją 3D}, Wydawnictwo naukowe PWN, Warszawa 2010

\bibitem{bib:gpu_gems} Lars Nyland, Mark Harris, Jan Prins, \emph{GPU Gems 3}, Chapter 31. Fast N-Body Simulation with CUDA, [online] \url{http://http.developer.nvidia.com/GPUGems3/gpugems3_ch31.html} [dostęp: 01-01-2017]

\end{thebibliography}

\end{document}